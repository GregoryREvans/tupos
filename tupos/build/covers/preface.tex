\documentclass[11pt]{article}
\usepackage{fontspec}
\usepackage[utf8]{inputenc}
\usepackage{xunicode}
\setmainfont{Bell MT}
\usepackage[paperwidth=17in,paperheight=11in,margin=1in,headheight=0.0in,footskip=0.5in,includehead,includefoot,portrait]{geometry}
\usepackage[absolute]{textpos}
\TPGrid[0.5in, 0.25in]{23}{24}
\parindent=0pt
\parskip=12pt
\usepackage{nopageno}
\usepackage{graphicx}
\graphicspath{ {./images/} }
\usepackage{amsmath}
\usepackage{tikz}
\usepackage{multicol}
\newcommand*\circled[1]{\tikz[baseline=(char.base)]{
            \node[shape=circle,draw,inner sep=1pt] (char) {#1};}}

\begin{document}


\begin{center}
\huge FOREWORD
\end{center}

\begingroup

\begin{center}
\leftskip3.25in
\underline{Thaumaturgike}, is that Art Mathematicall, which giueth certaine order
to make straunge workes, of the sense to be perceiued, and of men
greatly to be wondred at. By sundry meanes, this \textit{Wonder-worke} is
wrought. Some, by Pneumatithmie. As the workes of Ctesibius and
Hero, Some by waight. wherof Timæus speaketh. Some, by Stringes
strayned, or Springs, therwith Imitating liuely Motions. Some, by other
meanes, as the Images of Mercurie: and the brasen hed, made by Albertus
Magnus, which dyd seme to speake. Boethius was excellent in these
feates. To whom, Cassiodorus writyng, sayth. Your purpose is to know
profound thynges: and to shew meruayles. By the disposition of your
Arte, Metals do low: Diomedes of brasse, doth blow a Trumpet loude:
a brasen Serpent hisseth: byrdes made, sing swetely. Small thynges we
rehearse of you, who can Imitate the heauen.
\rightskip\leftskip
\phantom{text} \hfill (The Mathematicall Praeface to Elements of Geometrie of Euclid of Megara -- John Dee)
\end{center}

\endgroup


\begin{center}
\huge PERFORMANCE NOTES
\end{center}

\begin{multicols}{2}

\leftskip0.25in
\includegraphics[height=1cm]{./images/signature.png} \pmb{Non-power-of-2 denominated time signatures} : In this work, some measures are durated by a time signature with a denominator which is not a power of 2. ``In each case, the principle applicable to the derivation of more conventional meters (understood as subdivisions of a whole note) is maintained. For example, $\frac{2}{10}$ signifies a bar composed of two beats, each of which is equal to one-tenth of a whole note.'' Just as in much contemporary music where the time signatures of compound meters do not prolate the contents of their measures (e.g. a quarter note in $\frac{6}{8}$ and in $\frac{3}{4}$ are fundamentally the same speed) so too do these unusual time signatures preserve the basic metronomic tempo. \\
\rightskip\leftskip
\phantom{text} \hfill \phantom{()}


\vspace*{0.25cm}

\leftskip0.25in
\includegraphics[height=1cm]{./images/air_tone.png} \pmb{Air Tone} : A tone color consisting of a mixture of air noise and pitch with more noise than pitch. \\
\rightskip\leftskip
\phantom{text} \hfill \phantom{()}


\vspace*{0.25cm}

\leftskip0.25in
\includegraphics[height=1cm]{./images/half_air_tone.png} \pmb{Half Air Tone} : A tone color consisting of a mixture of air noise and pitch with more pitch than noise. \\
\rightskip\leftskip
\phantom{text} \hfill \phantom{()}


\vspace*{0.25cm}

\leftskip0.25in
\includegraphics[height=1cm]{./images/frullato.png} \pmb{Frullato} : Flutter-tongue technique performed with either the tongue or the throat. No distinction is made. \\
\rightskip\leftskip
\phantom{text} \hfill \phantom{()}


\vspace*{0.25cm}

\leftskip0.25in
\includegraphics[height=1cm]{./images/key_click.png} \pmb{Key Clicks} : Noise produced by percussively slamming the flute keys against the body of the instrument, while also playing the written tone. When no breath is used, the note head is simply an ``X''. \\
\rightskip\leftskip
\phantom{text} \hfill \phantom{()}

\vspace*{0.25cm}

\leftskip0.25in
\includegraphics[height=1.2cm]{./images/pizzicato.png} \pmb{Pizzicato} : Articulation produced by a plosive ``P'' sound. \\
\rightskip\leftskip
\phantom{text} \hfill \phantom{()}

\vspace*{0.25cm}

\leftskip0.25in
\includegraphics[height=1.7cm]{./images/fingerings.png} \pmb{Bisbigliando} : Play fingerings \circled{1}, \circled{2}, \circled{3}, and \circled{4} as increasingly different versions of the pitches over which they appear. Choose fingerings that minimize differences in pitch while maximizing differences in color. \\
\rightskip\leftskip
\phantom{text} \hfill \phantom{()}


\vspace*{0.25cm}

\leftskip0.25in
\includegraphics[height=5cm]{./images/polyphony.png} \pmb{Polyphonic Structures} : Decoupled colorations of the basic sounding line. \\
\rightskip\leftskip
\phantom{text} \hfill \phantom{()}


\vspace*{0.25cm}

\leftskip0.25in
\includegraphics[height=3cm]{./images/interruption.png} \pmb{Interruptive Polyphony} : Curtail the duration of the current note at the start of a subsequent note in a parallel line, attempting as much as possible to delineate the independence of each line. These passages should be performed as ``imaginary'' polyphony or ``compound'' melody rather than one consolidated line. \\
\rightskip\leftskip
\phantom{text} \hfill \phantom{()}


\vspace*{0.25cm}

\leftskip0.25in
\includegraphics[height=0.5cm]{./images/angle.png} \pmb{Flute Angle} : Rotate the tube of the flute so that the angle between the lips and the aperture of the flute is represented by the graphic indication where the symbols $\supset$ and $\subset$ mean that the flute should be turned inwards and outwards respectively, as far as possible whilst still producing recognizable pitch; $\cup$ means normal position, and the angled sign means a position between the normal and either extreme. \\
\rightskip\leftskip
\phantom{text} \hfill \phantom{()}

\vspace*{0.25cm}

\leftskip0.25in
\includegraphics[height=0.5cm]{./images/embouchure_tension.png} \pmb{Embouchure Tension} : An empty box represents a loose (wide) embouchure and a solid box represents a tight embouchure. \\
\rightskip\leftskip
\phantom{text} \hfill \phantom{()}

\vspace*{0.25cm}

\leftskip0.25in
\includegraphics[height=0.8cm]{./images/multi_trill.png} \pmb{Trill Figures} : In multi-trills of two or three subordinate tones, the pitches are freely combined into trills of varying intervallic content, preferably in constant, irregular reorderings. \\
\rightskip\leftskip
\phantom{text} \hfill \phantom{()}

\vspace*{0.25cm}

\leftskip0.25in
\includegraphics[height=0.5cm]{./images/tongue.png} \pmb{Tongue Techniques} : Normal, exaggerated, and tongue-less articulation. \\
\rightskip\leftskip
\phantom{text} \hfill \phantom{()}

\vspace*{0.25cm}

\leftskip0.25in
\includegraphics[height=0.5cm]{./images/vibrato.png} \pmb{Vibrato} : Various graphic conventions are employed for indicating different speeds and intensity of vibrato. \\
\rightskip\leftskip
\phantom{text} \hfill \phantom{()}

\vspace*{0.25cm}

\leftskip0.25in
\includegraphics[height=0.5cm]{./images/smorzando.png} \pmb{Smorzando} : Clearly articulated smorzato, executed with movement of the lips so as to partially block and release sustained sounds. \\
\rightskip\leftskip
\phantom{text} \hfill \phantom{()}

\vspace*{0.25cm}

\leftskip0.25in
\includegraphics[height=0.5cm]{./images/articulated_microtones.png} \pmb{Splattered Microtonal Wave} : The triple-staccato represents a continuous, iterated tongue attack as fast as possble rather than a triple-tongue. This could possibly be performed with a frullato of the tongue against the back of the teeth. In most cases this notation is accompanied by a wavy line indicating arbitrary microtonal deviation from the written pitch. \\
\rightskip\leftskip
\phantom{text} \hfill \phantom{()}

\vspace*{0.25cm}


\leftskip0.25in
\pmb{Mouth} : Shaping the flute sounds with the mouth are notated in IPA notation. When vowels are present, these syllables should be voiced or half-sung at an arbitrary pitch, preferably in the modal register of the voice. \\
\rightskip\leftskip
\phantom{text} \hfill \phantom{()}

\vspace*{0.25cm}


\leftskip0.25in
\pmb{Accidentals} : After temporary accidentals, cancellation marks are printed also in the following measure (for notes in the same octave) and, in the same measure, for notes in other octaves, but they are printed again if the same note appears later in the same measure, except if the note is immediately repeated. \\
\rightskip\leftskip
\phantom{text} \hfill \phantom{()}

\end{multicols}

\vspace*{20\baselineskip}

\begin{center}
\textit{tupos} was composed for Joshua Paul Stine.
\end{center}

\vspace*{15\baselineskip}

\begin{center}
duration: c. 12'00''
\end{center}

%\vspace*{10\baselineskip}
.

\end{document}
